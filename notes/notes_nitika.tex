\documentclass[letterpaper, 10pt]{article}

\usepackage{amsmath}
\usepackage{float} 
\usepackage{graphicx}

\usepackage[margin=1in]{geometry}

\title{Radio Background Research Notes}
\author{Nitika Yadlapalli}
\date{}


\begin{document}
\maketitle

\section{Sky Brightness Model for a Disk + Halo Galactic Model}

\section{Deriving Emissivity from Brightness Temperature}
Given some brightness temperature for both the halo and the disk (such as those given in the Subrahmanyan and Cowsik paper), I want to calculate the the emissivity (power per volume) for the disk and halo. To start, we can use the brightness temperature to calculate specific intensity using the Rayleigh Jeans approximation. 
\[ I_{\nu} = \frac{2\nu^{2}}{c^{2}}kT \]

From the specific intensity, to derive a emissivity, we need to integrate over the solid angle and divide by the length of the line of sight. We can assume that the intensity is constant with respect to the $\phi$ and $\theta$ values in question. All values of brightness temperature in the Subrahmanyan and Cowsik paper are given based on an observer in the galactic center. In the following equation, $\Omega$ is the solid angle and \emph{s} is the length of the line of sight. This will be either $R_{halo}$ or $R_{disk}$.
\[ P_{\nu} = \frac{\Omega I_{\nu}}{s} \]

Given this, the emissivity for the spherical galactic halo is given by 

\[P_{\nu, halo} = (4\pi)\left(\frac{2\nu^{2}}{c^{2}}kT\right)\left(\frac{1}{R_{halo}}\right)\]

while the emissivity for a disk is given by 
\[P_{\nu, disk} = \left[ 2\pi tan^{-1}\left(\frac{h_{disk}}{R_{disk}}\right) \right]\left(\frac{2\nu^{2}}{c^{2}}kT\right)\left(\frac{1}{R_{halo}}\right)\]

Then, the flux density along a line of sight can be given by 

\[F_{\nu} = P_{\nu, halo}D_{halo} + P_{\nu, disk}D_{disk} \]

\end{document}